% XeLaTeX can use any Mac OS X font. See the setromanfont command below.
% Input to XeLaTeX is full Unicode, so Unicode characters can be typed directly into the source.

% The next lines tell TeXShop to typeset with xelatex, and to open and save the source with Unicode encoding.

%!TEX TS-program = xelatex
%!TEX encoding = UTF-8 Unicode

\documentclass[12pt]{article}
\usepackage{geometry}                % See geometry.pdf to learn the layout options. There are lots.
\geometry{letterpaper}                   % ... or a4paper or a5paper or ... 
%\geometry{landscape}                % Activate for for rotated page geometry
%\usepackage[parfill]{parskip}    % Activate to begin paragraphs with an empty line rather than an indent
\usepackage{graphicx}
\usepackage{amssymb}
\usepackage[linesnumberedhidden,ruled,boxed]{algorithm2e}

\usepackage{amsmath}
\usepackage{bm}
\usepackage{esdiff}
\usepackage{indentfirst}
\usepackage[hidelinks]{hyperref}
\usepackage[all]{hypcap}

\usepackage{vhistory}

%\usepackage{standalone}
\usepackage{tikz}

\usepackage{natbib}

\newcommand{\pluseq}{\mathrel{+}=}

\numberwithin{equation}{section}

\hypersetup{
    pdfinfo={
        Title={Note on Learning Neural Network},
        Author={Wang Jian},
        Subject={BP algorithm}
    }
}

\title{Note on Learning Neural Network}
\author{Wang Jian}
\date{Version \vhCurrentVersion\ from \vhCurrentDate}

\begin{document}
\maketitle
\begin{versionhistory}
  \vhEntry{0.1}{12/14/2014}{Wang Jian}{Basic backpropagation}
  \vhEntry{0.2}{01/11/2015}{Wang Jian}{RTRL for RNN}
   \vhEntry{0.3}{03/11/2015}{Wang Jian}{BPTT for RNN}
\end{versionhistory}

\tableofcontents
\newpage

\section{Preliminary}
\subsection{Non-linear function}
\subsubsection{Sigmoid}
\begin{equation}
    \sigma(x) = \frac{1}{1 + e^{-x}}
\end{equation}
\begin{equation} \label{eq:sigmoid_diff}
    \frac{d\sigma(x)}{dx} = \sigma(x)(1 - \sigma(x)) 
\end{equation}

\subsubsection{Hyperbolic tangent}
\begin{equation}
    \tanh(x) = \frac{e^x-e^{-x}}{e^x + e^{-x}}
\end{equation}
\begin{equation}
    \frac{d\tanh(x)}{dx} = 1 - \tanh^2(x)
\end{equation}

\subsubsection{Softmax}
\begin{equation} \label{eq:softmax}
    S_i(\bm{z}) = \frac{e^{z_i}}{\sum_k{e^{z_k}} }
\end{equation}
\begin{equation} \label{eq:softmax_diff}
    \frac{\partial\log(S_i(\bm{z}))}{\partial z_j} = \delta_{ij} - S_i({\bm{z}}) 
\end{equation}
where $\delta_{ij}$ is  Kronecker delta, 
$$
    \delta_{ij} = \left\{\begin{array}{rl}
                                    1 & i = j \\
                                    0 & i \neq j 
                                \end{array} \right.
$$

\subsection{Cross entropy}
\begin{equation} \label{eq:cross_ent}
    J(\bm{w}) = -\sum_k{t_k}{\log z_k}
\end{equation}

\subsection{Gradient descent}
In order to find a local minimum of a convex function, we can iteratively update the parameters using the following equation,
\begin{equation} \label{eq:gd}
    \Delta\bm{w} = -\alpha\frac{\partial J}{\partial \bm{w}}
\end{equation}
where, $\alpha$ is the learning rate, controlling step size of every iteration.

\subsubsection{Batch Gradient Descent}
For data set with many examples, we can apply \autoref{eq:gd} by simply adding up all examples,
\begin{equation} \label{eq:bgd}
    \Delta\bm{w} = -\alpha\sum_r\frac{\partial J^{r}}{\partial \bm{w}}
\end{equation}
where, $r$ runs through all examples in data set. The algorithm for BGD is shown in \autoref{alg:bgd}.

\begin{algorithm}[H] 
  \caption{Batch Gradient Descent} \label{alg:bgd}
  \KwIn{$N$: number of iterations, $R$: number of examples}
  
  \For{$n = 1:N$} {
    $\Delta\bm{w} = 0$ \\
    \For{$r = 1:R$} {
       $\Delta\bm{w} \pluseq -\alpha\frac{\partial J^{r}}{\partial \bm{w}}$ 
    }
    $\bm{w} \pluseq \Delta\bm{w}$
  }
\end{algorithm}

\subsubsection{Stochastic Gradient Descent}
BGD have to run through all the samples to do a single update, for large data set this may cause problems.
To make the updating faster, we can do the update with every one example, which leads to the SGD algorithm, shown in \autoref{alg:sgd}.

\begin{algorithm}[H] 
  \caption{Stochastic Gradient Descent} \label{alg:sgd}
  \KwIn{$N$: number of iterations, $R$: set of examples}
  
  \For{$n = 1:N$} {
    \For{$r = UniqRandom(R)$} {
       $\Delta\bm{w} = -\alpha\frac{\partial J^{r}}{\partial \bm{w}}$ \\
       $\bm{w} \pluseq \Delta\bm{w}$
    }
  }
\end{algorithm}
where the $UniqRandom$ function randomly and not repeatedly chooses one example from data set. 

SGD often converges much faster compared to BGD but the error function is not as well minimised as in the case of BGD. 
To compromise, we can use a method called Mini-Batch SGD, which uses a small subset instead the whole training set to do the update, shown in \autoref{alg:mini}.

\begin{algorithm}[H] 
  \caption{Mini-Batch Stochastic Gradient Descent} \label{alg:mini}
  \KwIn{$N$: number of iterations, $B$: set of mini-batch, $R$: set of examples}
  
  \For{$n = 1:N$} {
     \For{$B = UniqRandomSet(R)$} {
       $\Delta\bm{w} = 0$ \\
       \For{$r = UniqRandom(B)$} {
         $\Delta\bm{w} \pluseq -\alpha\frac{\partial J^{r}}{\partial \bm{w}}$ \\
       }
       $\bm{w} \pluseq \Delta\bm{w}$
    }
  }
\end{algorithm}

As we can see, batch gradient descent is deterministic, which means that every time you run BGD for a given training set, 
you will get the same optimum in the same number of iterations.  
Stochastic gradient descent is, however, stochastic.  Because you are no longer using your entire training set a once, 
and instead picking one or more examples at a time in some likely random fashion, 
each time you tun SGD you will obtain a different optimum and a unique cost vs. iteration history.

\subsection{The Multivariable Chain Rule}
Multivariable Chain Rules allow us to differentiate with respect to any of the variables involved.
One can use the \textit{variable-dependence diagram} as a simple way to apply this Chain Rule. 
we can compute the derivatives by 
simply adding up all paths starting at the dependent variables and ending at the independent variables, 
multiplying derivatives along each path.

For example, suppose that we have $x=f_1(u)$, $x=f_2(v)$,$y=f_3(u)$,$y=f_4(v)$,$z=f_5(x)$ and $z=f_6(y)$,
\autoref{fig:chain_rule} shows the diagram,
\begin{figure}[ht]
\begin{center}
  %\includestandalone{chain_rule}
  \begin{tikzpicture}
    \node (z) at ( 0,  0) {$z$}; 
    \node (x) at (-2, -2) {$x$};
    \node (y) at ( 2, -2) {$y$};
    \node (u) at ( -1.5, -4) {$u$};
    \node (v) at ( 1.5, -4) {$v$};

    \begin{scope}[every path/.style={->}]
       \draw (u) -- node [left] {$\frac{\partial x}{\partial u}$} (x);
       \draw (u) -- node [near end, above] {$\frac{\partial y}{\partial u}$} (y); 
       \draw (v) -- node [near end, above] {$\frac{\partial x}{\partial v}$} (x);
       \draw (v) -- node [right] {$\frac{\partial y}{\partial v}$} (y); 
       \draw (x) -- node [left] {$\frac{\partial z}{\partial x}$} (z);
       \draw (y) -- node [right] {$\frac{\partial z}{\partial y}$} (z);
    \end{scope}  
\end{tikzpicture}


  
  \caption{Variable-dependence diagram}
  \label{fig:chain_rule}
\end{center}
\end{figure}

Then we can get the derivatives,
$$
    \frac{\partial z}{\partial u} = \frac{\partial z}{\partial x}\frac{\partial x}{\partial u} +  \frac{\partial z}{\partial y}\frac{\partial y}{\partial u}
$$
$$
    \frac{\partial z}{\partial v} = \frac{\partial z}{\partial x}\frac{\partial x}{\partial v} +  \frac{\partial z}{\partial y}\frac{\partial y}{\partial v}
$$

\subsection{Network architecture}
For the network discussed in this paper, the \textit{activation function} of input and hidden layer is sigmoid or hyperbolic tangent,
the activation function of output layer is softmax, and the \textit{loss function} is cross entropy.

The target vector is 1-of-V coding, i.e.
$$
    t_k = \left\{\begin{array}{rl}
                                    1 & k = k_0 \\
                                    0 & k \neq k_0 
                                \end{array} \right.
$$
then the loss function \eqref{eq:cross_ent} becomes:
\begin{equation}  \label{eq:loss}
    J(\bm{w}) = -\sum_k t_k\log z_k = -\log z_{k_0}
\end{equation}

Throughout this paper, $x_i$ denotes the input, $y_j$ denotes hidden layer activation,
$z_k$ denotes output layer activation. Moreover, $I$ is the input layer size ant $\mathcal{I}$ is the set of indices for input layer, 
$H$ is the hidden layer size and $\mathcal{H}$ is the set of indices for hidden layer,
$O$ is the output layer size  and $\mathcal{O}$ is the set of indices for output layer. 
To simplify notations, whenever involving summation, $i$, $j$ and $k$ always
runs through the proper set.

\section{Feed-forward Network} \label{sec:ffnn}
A \textit{Feed-forward Neural Network}\cite{duda2012pattern} is an artificial neural network where connections between the units do not form a directed cycle. 
\subsection{Forward pass}
Input layer to hidden layer:
\begin{equation}
    net^h_j = \sum_i{w_{ji}x_i}
\end{equation}
\begin{equation} \label{eq:hidden_ac}
    y_j = \sigma(net^h_j)
\end{equation}

Hidden layer to output layer:
\begin{equation} \label{eq:ff_netk}
    net^o_k = \sum_j{w_{kj}y_j}
\end{equation}
\begin{equation} \label{eq:output_ac}
    z_k = S_k(\bm{net^o})
\end{equation}

\subsection{Backpropagation}
\subsubsection{Weight between hidden layer and output layer} 
The variable-dependence diagram for differentiating $J$ w.r.t. $w_{kj}$ is shown in \autoref{fig:chain_rule_output},
\begin{figure}[ht]
\begin{center}
  \begin{tikzpicture}
    \node (J) at ( 0,  0) {$J$}; 
    \node (netk) at ( 0, -2) {$net^o_k$};
    \node (wkj) at ( 0, -4) {$w_{kj}$};

    \begin{scope}[every path/.style={->}]
       \draw (J) -- node [left] {$\frac{\partial J}{\partial net^o_k}$} (netk);
       \draw (netk) -- node [left] {$\frac{\partial net^o_k}{\partial w_{kj}}$} (wkj);
    \end{scope}  
\end{tikzpicture}


  
  \caption{Variable-dependence diagram for $w_{kj}$}
  \label{fig:chain_rule_output}
\end{center}
\end{figure}

So the derivative of $w_{kj}$ is
\begin{equation}
    \frac{\partial J}{\partial w_{kj}} = \frac{\partial J}{\partial net^o_k}\frac{\partial net^o_k}{\partial w_{kj}}
\end{equation}
for the first term on the right side:
\begin{align} \label{eq:output_diff}
    \frac{\partial J}{\partial net^o_k} &= \frac{\partial( -\log z_{k_0})}{\partial net^o_k}  =   \frac{\partial( -\log S_{k_0}({\bm{net^o}}) )}{\partial net^o_k}     
                                                                                                                                         && \text{substituting \eqref{eq:loss} and \eqref{eq:output_ac}} \nonumber \\
                                                   &= \delta_{kk_0} - S_{k_0}({\bm{net^o}})                       && \text{from \eqref{eq:softmax_diff}} \nonumber \\
                                                   &= \delta_{kk_0} - z_{k_0}                                             && \text{from \eqref{eq:output_ac}}
\end{align}
and the second term can be deduced by differentiating \eqref{eq:ff_netk} :
\begin{equation}
    \frac{\partial net^o_k}{\partial w_{kj}} = y_j
\end{equation}

We can define the \textit{error} of output layer is
\begin{equation} \label{eq:output_err}
    e_k \equiv \delta_{kk_0} - z_{k_0}
\end{equation}
then, we get
\begin{equation}\label{eq:output_grad}
    \frac{\partial J}{\partial w_{kj}} = e_k y_j
\end{equation}


\subsubsection{Weight between input layer and hidden layer}
\autoref{fig:chain_rule_hidden} shows the variable-dependence diagram for differentiating $J$ w.r.t. $w_{ji}$,
\begin{figure}[ht]
\begin{center}
  \begin{tikzpicture}
    \node (J) at ( 0,  0) {$J$}; 
    \node (netk1) at ( -4, -2.5) {$net^o_1$};
    \node (netk1k) at ( -2, -2.5) {$\cdots$};
    \node (netk) at ( 0, -2.5) {$net^o_k$};
    \node (netkkn) at ( 2, -2.5) {$\cdots$};
    \node (netkn) at ( 4, -2.5) {$net^o_O$};
    \node (yj) at ( 0, -5) {$y_j$};
    \node (netj) at ( 0, -7) {$net^h_j$};
    \node (wji) at ( 0, -9) {$w_{ji}$};
    
    \begin{scope}[every path/.style={->}]
       \draw (J) -- node [above=.5em, left] {$\frac{\partial J}{\partial net^o_1}$} (netk1);
       \draw (J) -- node [left] {$\frac{\partial J}{\partial net^o_k}$} (netk);
       \draw (J) -- node [above=.5em, right] {$\frac{\partial J}{\partial net^o_O}$} (netkn);
       
       \draw (netk1) -- node [below=.6em, left] {$\frac{\partial net^o_1}{\partial y_j}$} (yj);
       \draw (netk) -- node [left] {$\frac{\partial net^o_k}{\partial y_j}$} (yj);
       \draw (netkn) -- node [below=.6em, right] {$\frac{\partial net^o_O}{\partial y_j}$} (yj);
       
       \draw (yj) -- node [left] {$\frac{\partial y_j}{\partial net^h_j}$} (netj);
       \draw (netj) -- node [left] {$\frac{\partial net^h_j}{\partial w_{ji}}$} (wji);
    \end{scope}  
\end{tikzpicture}


  
  \caption{Variable-dependence diagram for $w_{ji}$}
  \label{fig:chain_rule_hidden}
\end{center}
\end{figure}

The derivative of $w_{ji}$ is
\begin{equation} \label{eq:hidden_diff_w}
    \frac{\partial J}{\partial w_{ji}} = \frac{\partial J}{\partial y_j}\frac{\partial y_j}{\partial w_{ji}} 
                                             = \frac{\partial J}{\partial y_j}\frac{\partial y_j}{\partial net^h_j}\frac{\partial net^h_j}{\partial w_{ji}}
\end{equation}
where, $\frac{\partial J}{\partial y_j}$ is
\begin{align} \label{eq:hidden_diff}
    \frac{\partial J}{\partial y_j} &= \sum_k{\frac{\partial J}{\partial net^o_k}\frac{\partial net^o_k}{\partial y_j}}   && \nonumber \\
                                               &=   \sum_k{(\delta_{kk_0} - z_{k_0})\frac{\partial net^o_k}{\partial y_j}}     && \text{substituting \eqref{eq:output_diff}} \nonumber \\
                                               &=   \sum_k{(\delta_{kk_0} - z_{k_0})w_{kj}}    && \text{take partial derivative} \nonumber \\
                                               &=   \sum_k{e_k w_{kj}}   &&\text{substituting \eqref{eq:output_err}} 
\end{align}
make use of \eqref{eq:sigmoid_diff} and \eqref{eq:hidden_ac}
\begin{equation} \label{eq:hidden_diff_net}
   \frac{\partial y_j}{\partial net^h_j} = \sigma'(net^h_j) =  y_j(1 - y_j)
\end{equation}
and
\begin{equation} \label{eq:hidden_net_diff_w}
   \frac{\partial net^h_j}{\partial w_{ji}} = x_i
\end{equation}

Similarly, we define the error of hidden layer is
\begin{equation}
    e_j \equiv \sigma'(net^h_j) \sum_k{e_k w_{kj}}
\end{equation}
so that the gradient of $w_{ji}$ takes the same form as $w_{kj}$,
\begin{equation}
    \frac{\partial J}{\partial w_{ji}} = e_j x_i 
\end{equation}

Thus, for the non-output layer, calculating weight update involves: a) the input $x_i$; b) the derivative of output $\sigma'(net^h_j)$; 
and c) the error back propagating from upper layer $\sum_k{e_k w_{kj}} $.  

\section{Recurrent Neural Network}
\textit{Recurrent Neural Network}\cite{mikolov2012statistical} is a neural network where connections between hidden layer units form directed cycle. 
\subsection{Forward pass}
Input layer to hidden layer:
\begin{equation}
    net^h_j(t) = \sum_i{w_{ji}x_i(t)}
\end{equation}

Hidden layer to hidden layer:
\begin{equation} \label{eq:rnn_hidden_net}
    net^r_j(t) = \sum_{s \in \mathcal{S}}{w_{js}y_s(t - 1)}
\end{equation}
where we denote the new set of indices for hidden-to-hidden weights as $\mathcal{S}$.

Hidden layer activiation:
\begin{equation} 
    net^s_j(t) = net^h_j(t) + net^r_j(t)
\end{equation}
\begin{equation} \label{eq:rnn_hidden_ac}
    y_j(t) = \sigma(net^s_j(t))
\end{equation}

Hidden layer to output layer:
\begin{equation} \label{eq:rnn_ff_netk}
    net^o_k(t) = \sum_j{w_{kj}y_j(t)}
\end{equation}
\begin{equation} \label{eq:rnn_output_ac}
    z_k(t) = S_k(\bm{net^o}(t))
\end{equation}

\subsection{Backpropagation}
Gradient for $w_{kj}$ is essentially the  same as Feed-forward Network, i.e.
\begin{equation}
    \frac{\partial J}{\partial w_{kj}} = e_ky_j
\end{equation}

For $w_{ji}$ and $w_{js}$, there are several methods to calculate their gradients.

\subsubsection{Real-Time Recurrent Learning}
RTRL\cite{williams1989learning} computes the derivatives in an online way, thus the name "Real-Time".
Because of the recurrent term \eqref{eq:rnn_hidden_net}, variations in $w_{ji}$ give rise to variations in the error function through variations in the all $y_j$s.

The variable-dependence diagram for differentiating $J$ w.r.t. $w_{ji}$ is shown in \autoref{fig:chain_rule_rtrl_J_y},
\begin{figure}[ht]
\begin{center}
  \begin{tikzpicture}
    \node (J) at ( 0,  0) {$J(t)$}; 
    
    \node (netk1) at ( -4, -2.5) {$net^o_1(t)$};
    \node (netk1k) at ( -2, -2.5) {$\cdots$};
    \node (netk) at ( 0, -2.5) {$net^o_k(t)$};
    \node (netkkn) at ( 2, -2.5) {$\cdots$};
    \node (netkn) at ( 4, -2.5) {$net^o_O(t)$};
   
    \node (y1) at ( -4, -5) {$y_1(t)$}; 
    \node (y1j) at ( -2, -5) {$\cdots$};
    \node (yj) at ( 0, -5) {$y_j(t)$};
    \node (yjH) at ( 2, -5) {$\cdots$};
    \node (yH) at ( 4, -5) {$y_H(t)$};
    
     \node (wji) at ( 0, -7) {$w_{ji}$};
    
    \begin{scope}[every path/.style={->}]
       \draw (J) -- node [above=.5em, left] {$\frac{\partial J(t)}{\partial net^o_1(t)}$} (netk1);
       \draw (J) -- node [left] {$\frac{\partial J(t)}{\partial net^o_k(t)}$} (netk);
       \draw (J) -- node [above=.5em, right] {$\frac{\partial J(t)}{\partial net^o_O(t)}$} (netkn);
       
       \draw (netk1) -- node [right] {$\frac{\partial net^o_k(t)}{\partial y_j(t)}$} (y1);
       \draw (netk) --  (y1);
       \draw (netkn) -- (y1);
       
       \draw (netk1) -- (yj);
       \draw (netk) -- (yj);
       \draw (netkn) -- (yj);
       
       \draw (netk1) --  (yH);
       \draw (netk) -- (yH);
       \draw (netkn) -- node [left] {$\frac{\partial net^o_k(t)}{\partial y_j(t)}$} (yH);
       
       \draw[dashed] (y1) -- node [solid, below=.7em,  left] {$\frac{\partial y_1(t)}{\partial w_{ji}}$}  (wji);
       \draw[dashed] (yj) -- node [solid, left] {$\frac{\partial y_j(t)}{\partial w_{ji}}$}  (wji);
       \draw[dashed] (yH) -- node [solid, below=.7em, right] {$\frac{\partial y_H(t)}{\partial w_{ji}}$}  (wji);
    \end{scope} 
\end{tikzpicture}


  
  \caption{Variable-dependence diagram for $w_{ji}$ of RTRL. Dashed line denotes a path which may through several variables.}
  \label{fig:chain_rule_rtrl_J_y}
\end{center}
\end{figure}

We can write the derivative of $J$ w.r.t. $w_{ji}$, 
\begin{equation}
    \frac{\partial J}{\partial w_{ji}} = \sum_k{\frac{\partial J}{\partial net^o_k} \sum_{l \in \mathcal{S}}{\frac{\partial net^o_k}{\partial y_l(t)} \frac{\partial y_l(t)}{\partial w_{ji}}}}
\end{equation}

As for feed-forward network, $\frac{\partial J}{\partial net^o_k} = e_k$ and $\frac{\partial net^o_k}{\partial y_l(t)} = w_{kl}$.
For the $\frac{\partial y_l(t)}{\partial w_{ji}}$ part, the variable-dependence diagram is shown in \autoref{fig:chain_rule_rtrl_y_w},
\begin{figure}[ht]
\begin{center}
  \begin{tikzpicture}
    \node (yj) at ( 0, 0) {$y_j(t)$};
    \node (sj) at ( 0, -2) {$net^s_j(t)$};
    \node (netj) at ( -4, -4) {$net^h_j(t)$};
    \node (netrj) at ( 4, -4) {$net^r_j(t)$};
    \node (wji) at ( 0, -9) {$w_{ji}$};
    
    \node (y1) at ( 0, -6.5) {$y_1(t-1)$};
    \node (y1m) at ( 2, -6.5) {$\cdots$};
    \node (ym) at ( 4, -6.5) {$y_m(t-1)$};
    \node (ymh) at ( 6, -6.5) {$\cdots$};
    \node (yh) at ( 8, -6.5) {$y_H(t-1)$};
    
    \begin{scope}[every path/.style={->}]
       \draw (yj) -- node [left] {$\frac{\partial y_j(t)}{\partial net^s_j(t)}$} (sj);
       \draw (sj) -- node [above=.5em, left] {$\frac{\partial net^s_j(t)}{\partial net^h_j(t)}$} (netj);
       \draw (netj) -- node [left] {$\frac{\partial net^h_j(t)}{\partial w_{ji}}$} (wji);
       
       \draw (sj) -- node [above=.5em, right] {$\frac{\partial net^s_j(t)}{\partial net^r_j(t)}$} (netrj);
       \draw (netrj) -- node [above=.5em, left] {$\frac{\partial net^r_j(t)}{\partial y_1(t-1)}$} (y1);
       \draw (netrj) -- node [left] {$\frac{\partial net^r_j(t)}{\partial y_m(t-1)}$} (ym);
       \draw (netrj) -- node [above=.5em, right] {$\frac{\partial net^r_j(t)}{\partial y_H(t-1)}$} (yh);

       \draw[dashed] (y1) -- node [solid, near start, left] {$\frac{\partial y_1(t-1)}{\partial w_{ji}}$}  (wji);
       \draw[dashed] (ym) -- node [solid, left] {$\frac{\partial y_m(t-1)}{\partial w_{ji}}$}  (wji);
       \draw[dashed] (yh) -- node [solid, below=.7em, right] {$\frac{\partial y_H(t-1)}{\partial w_{ji}}$}  (wji);
    \end{scope} 
\end{tikzpicture}


  
  \caption{Variable-dependence diagram for $\frac{\partial y_l(t)}{\partial w_{ji}}$ part of RTRL, where $l = j$.}
  \label{fig:chain_rule_rtrl_y_w}
\end{center}
\end{figure}

\autoref{fig:chain_rule_rtrl_y_w} shows the situation where $l = j$. The diagram of $l \neq j$ does not have the left-most path from $net^s_j(t)$ to $w_{ji}$.
Therefore, the derivative of $y_l(t)$ w.r.t. $w_{ji}$ is,
\begin{equation}
    \frac{\partial y_l(t)}{\partial w_{ji}} = \delta_{lj}\frac{\partial y_l(t)}{\partial net^s_l(t)} \frac{\partial net^s_l(t)}{\partial net^h_l(t)}\frac{\partial net^h_l(t)}{\partial w_{ji}} 
                                                           + \frac{\partial y_l(t)}{\partial net^s_l(t)} \frac{\partial net^s_l(t)}{\partial net^r_l(t)} 
                                                              \sum_{m \in \mathcal{S}} \frac{\partial net^r_l(t)}{\partial y_m(t-1)} \frac{\partial y_m(t-1)}{\partial w_{ji}}
\end{equation}
Note that, $\frac{\partial net^s_l(t)}{net^h_l(t)} = \frac{\partial net^s_l(t)}{net^r_l(t)} = 1$, 
so the first term of right side is the same as \eqref{eq:hidden_diff_net} and \eqref{eq:hidden_net_diff_w},
\begin{equation} \label{eq:rnn_hidden_diff_w_i}
    \frac{\partial y_l(t)}{\partial net^s_l(t)} \frac{\partial net^s_l(t)}{\partial net^h_l(t)}\frac{\partial net^h_l(t)}{\partial w_{ji}} = \sigma'(net^s_l(t))x_i(t)
\end{equation}
However, the second term involves recurrent variable,
\begin{equation} \label{eq:rnn_hidden_diff_w_h}
   \frac{\partial y_l(t)}{\partial net^s_l(t)} \frac{\partial net^s_l(t)}{\partial net^r_l(t)} 
    \sum_{m \in \mathcal{S}} \frac{\partial net^r_l(t)}{\partial y_m(t-1)} \frac{\partial y_m(t-1)}{\partial w_{ji}}  
             = \sigma'(net^s_l(t))\sum_{m \in \mathcal{S}} w_{jm} \frac{\partial y_m(t-1)}{\partial w_{ji}}
\end{equation}

thus, combining  \eqref{eq:rnn_hidden_diff_w_i} and \eqref{eq:rnn_hidden_diff_w_h}, we get,
\begin{equation}
    \frac{\partial y_l(t)}{\partial w_{ji}} = \sigma'(net^s_l(t))(\delta_{lj}x_i(t) + \sum_{m \in \mathcal{S}} w_{jm} \frac{\partial y_m(t-1)}{\partial w_{ji}})
\end{equation}

Put all together, we get the final result,
\begin{equation} \label{eq:rnn_wji}
    \frac{\partial J}{\partial w_{ji}} = \sum_k{e_k \sum_{l \in \mathcal{S}} { w_{kl} \sigma'(net^s_l(t))(\delta_{lj}x_i(t) + \sum_{m \in \mathcal{S}} w_{jm} \frac{\partial y_m(t-1)}{\partial w_{ji}})}}
\end{equation}

The hidden to hidden layer weights $w_{js}$ is similar with $w_{ji}$, but the variable-dependence diagram for $\frac{\partial y_l(t)}{\partial w_{js}}$ 
also does not have the left-most $net^h_j$-through path in \autoref{fig:chain_rule_rtrl_y_w}. i.e.,
\begin{equation}
    \frac{\partial y_l(t)}{\partial w_{js}} =  \frac{\partial y_l(t)}{\partial net^s_l(t)} \frac{\partial net^s_l(t)}{\partial net^r_l(t)} 
                                                               \frac{\partial net^r_l(t)}{\partial w_{js}}
\end{equation}

Next, From \eqref{eq:rnn_hidden_net} we can get $\frac{\partial net^r_l(t)}{\partial w_{js}}$,
\begin{equation}
    \frac{\partial net^r_l(t)}{\partial w_{js}} = \delta_{lj} y_s(t-1) +
                                                               \sum_{m \in \mathcal{S}} {w_{jm}\frac{\partial y_m(t-1)}{\partial w_{js}}}
\end{equation}

Therefore, we get the derivative of $w_{js}$,
\begin{equation} \label{eq:rnn_wjs}
    \frac{\partial J}{\partial w_{js}} = \sum_k{e_k \sum_{l \in \mathcal{S}} { w_{kl} \sigma'(net^s_l(t))(\delta_{lj}y_s(t-1) 
                                                      + \sum_{m \in \mathcal{S}} w_{jm} \frac{\partial y_m(t-1)}{\partial w_{js}})}}
\end{equation}

Following \cite{williams1989learning}, we can combine \eqref{eq:rnn_wji} and \eqref{eq:rnn_wjs}, if we define
$$
    p_n(t) = \left\{\begin{array}{ll}
                                    x_n(t) &  n \in \mathcal{I}  \\
                                    y_n(t-1) &  n \in \mathcal{S} 
                          \end{array} \right.
$$
where $n \in \mathcal{I} \cup \mathcal{S}$, then we can get a unified form of result
\footnote{The result in the original paper \cite{williams1989learning} seems simpler than ours, 
this is because the network in their paper does not have a output layer.},
\begin{equation}
    \frac{\partial J}{\partial w_{jn}} = \sum_k{e_k \sum_{l \in \mathcal{S}} { w_{kl} \sigma'(net^s_l(t))(\delta_{lj}p_n(t) 
                                                      + \sum_{m \in \mathcal{S}} w_{jm} \frac{\partial y_m(t-1)}{\partial w_{jn}})}}
\end{equation}

\subsubsection{Backpropagation Through Time}
BPTT\cite{Williams90anefficient} can be derived by unfolding the temporal operation of a network into a 
multilayer feedforward network that grows by one layer on each time step. In detail, in every time step, we
performs followings sequentially,as shown in \autoref{fig:bptt},
\begin{enumerate}
\item In the forward pass, we add one hidden layer, connecting from input layer and hidden layer for time $t-1$,
         then compute the activation of the new layer as activation of time $t$;
\item In the backpropagation, we compute the derivatives using BP through the whole network at time $t$, 
         fixing all weights the same, then update the weights simultaneously.
\end{enumerate}
Note that each hidden layer in the unfolded network corresponds a single time step, thus when we
propagate errors through layers, we are actually ``backpropagating through time".

\begin{figure}[ht]
\begin{center}
  \begin{tikzpicture}
    \begin{scope}[every path/.style={thick, >=latex, ->}]
       \pgfmathsetmacro{\ts}{3} % time steps
      
       \pgfmathsetmacro{\hht}{2}  % hidden layer height
       \pgfmathsetmacro{\iht}{4} % input and output layer height
       \pgfmathsetmacro{\wid}{0.5} % layer width
       
       \pgfmathsetmacro{\ihd}{0.5} % distance between input and hidden layer
       \pgfmathsetmacro{\tw}{2} % width between time steps
       
       \pgfmathsetmacro{\th}{\hht + \ihd + \iht / 2 - \hht / 2} % height between time steps
       
       \pgfmathsetmacro{\ar}{\wid * 1.5} % length of input and output arrow
       \pgfmathsetmacro{\iap}{4 / 5} % position of input arrow
       \pgfmathsetmacro{\oa}{8} % num of output arrow
       \pgfmathsetmacro{\oad}{\iht / \oa} % distance between output arrow
       
       \pgfmathsetmacro{\wtd}{.2} % distances between weight lines
        
       \foreach \t in {1,...,\ts} {
         \pgfmathsetmacro{\x}{\t*(\wid + \tw)}
         \pgfmathsetmacro{\h}{\t*\th}
         
         \draw (\x, \h) rectangle +(\wid, \hht);
         \pgfmathsetmacro{\y}{\ts - \t + 1}
         
         \pgfmathparse{(\y==0)?1:0}
         \ifdim\pgfmathresult pt>0pt {%
           \path (\x, \h) --  +(\wid, 0) node[midway, below] {$y(t)$};
         } \else {%
           \path (\x, \h) --  +(\wid, 0) node[midway, below] {$y(t-\pgfmathprintnumber{\y})$};
         } %
         \fi
         
         \pgfmathsetmacro{\h}{\t*\th + \hht + \ihd}
         \draw (\x, \h) rectangle +(\wid, \iht);
         \pgfmathsetmacro{\y}{\ts - \t}
         
         \pgfmathparse{(\y==0)?1:0}
         \ifdim\pgfmathresult pt>0pt {%
           \path (\x, {\h + \iht}) --  +(\wid, 0) node[midway, above] {$x(t)$};
         } \else {%
           \path (\x, {\h + \iht}) --  +(\wid, 0) node[midway, above] {$x(t-\pgfmathprintnumber{\y})$};
         } %
         \fi
         \draw ({\x - \ar}, {\h + (\iap * \iht)}) -- +(\ar, 0);

         \draw ({\x + \wid}, {\h + \iht / 2}) -- node[midway, above]{$w_{ji}$} +(\tw, 0);
         \draw ({\x + \wid + \tw}, {\h + \iht / 2 - \wtd}) [dashed] -- +(-\tw, 0);
         \draw ({\x + \wid}, {\t*\th + \hht / 2}) -- node[midway, left]{$w_{js}$} +(\tw, {\hht / 2 + \ihd + (\iht / 2 - \hht / 2) + \hht / 4});
         \draw ({\x + \wid + \tw}, {\t*\th + \hht / 2 + \hht / 2 + \ihd + (\iht / 2 - \hht / 2) + \hht / 4 - \wtd}) [dashdotted] -- ({\x + \wid}, {\t*\th + \hht / 2 - \wtd});
       }
       
        \pgfmathsetmacro{\x}{(\ts + 1)*(\wid + \tw)}
        \pgfmathsetmacro{\h}{(\ts + 1)*\th}
        \draw (\x, \h) rectangle +(\wid, \hht);
        \path (\x, \h) --  +(\wid, 0) node[midway, below] {$y(t)$};
        
        \pgfmathsetmacro{\x}{(\ts + 1)*(\wid + \tw) + \wid}
        \pgfmathsetmacro{\h}{(\ts + 1)*\th + \hht / 2}
        \draw (\x, \h) -- +(\tw, 0) node[midway, above] {$w_{kj}$};
        \draw ({\x + \tw}, {\h - \wtd}) [dotted] -- +(-\tw, 0);
         
        \pgfmathsetmacro{\x}{(\ts + 1)*(\wid + \tw) + \wid + \tw}
        \pgfmathsetmacro{\h}{(\ts + 1)*\th + \hht / 2 - \iht / 2}
        \draw (\x, \h) rectangle +(\wid, \iht);
        \path (\x, {\h + \iht}) --  +(\wid, 0) node[midway, above] {$z(t)$};
        
        \pgfmathsetmacro{\x}{(\ts + 1)*(\wid + \tw) + \wid + \tw + \wid}
        \foreach \i in {1,...,\oa} {
          \draw (\x, {\h + 0.5 * \oad + (\i - 1) * \oad}) -- +(\ar, 0);
        }
    \end{scope}  
\end{tikzpicture}


  
  \caption{RNN unfolded as a feed-forward NN, here 3 time steps back in time. 
                Dashed and dotted arrows indicate how the gradients are propagated}
  \label{fig:bptt}
\end{center}
\end{figure}

The unfold network is simply a feed-forward network, thus we can directly use the results obtained in \autoref{sec:ffnn}.
 At time $t_0$, we first compute the errors for each layer,
\begin{equation}
      e_j(t) = \left\{\begin{array}{ll}
                                    \sum_k{e_k(t) w_{kj}} &  t=t_0  \\
                                    \sum_{l \in \mathcal{S}}{e_l(t + 1) w_{lj}} &  t < t_0 
                          \end{array} \right.
\end{equation}

Gradient of $w_{kj}$ is the same as \eqref{eq:output_grad}.
For $w_{ji}(t)$, denoting the weight comes out from $x_i(t)$, we have
\begin{equation}
    \frac{\partial J}{\partial w_{ji}(t)} = x_i(t)e_j(t)
\end{equation}

Because that we'd like to update the weights simultaneously in one time step, we add up all $t_0$-layer gradients to compute the $\Delta\bm{w}$,
\begin{equation}\label{eq:bptt_delta_wji}
   \Delta w_{ji}(t_0) = -\alpha\sum_{t'=1}^{t_0}\frac{\partial J}{\partial w_{ji}(t')} = -\alpha\sum_{t'=1}^{t_0}x_i(t')e_j(t')
\end{equation}

Similarly, the updating value of $w_{js}$ is 
\begin{equation}\label{eq:bptt_delta_wjs}
   \Delta w_{js}(t_0) = -\alpha\sum_{t'=1}^{t_0}\frac{\partial J}{\partial w_{js}(t')} = -\alpha\sum_{t'=1}^{t_0}y_s(t'-1)e_j(t')
\end{equation}

It is easy to see that back propagate for the whole time history for every time step is too complexity, and the error gradients quickly vanish or explode).
So several steps of unfolding are sufficient, which leads to the \emph{Truncated BPTT}. Suppose we only unfold for $T$ time step, 
then \eqref{eq:bptt_delta_wji} and \eqref{eq:bptt_delta_wjs} become
\begin{equation}
   \Delta w_{ji}(t_0) = -\alpha\sum_{t'=t_0-T}^{t_0}x_i(t')e_j(t')
\end{equation}

and

\begin{equation}
   \Delta w_{js}(t_0) =  -\alpha\sum_{t'=t_0-T}^{t_0}y_s(t'-1)e_j(t')
\end{equation}


As the on-line update would lead to large computational complexity, we can update the weights in mini-batches. 
The flow of gradients is illustrated in \autoref{fig:bptt_block}.

\begin{figure}[ht]
\begin{center}
  \begin{tikzpicture}
    \begin{scope}[every path/.style={thick, >=latex, ->}]
       \pgfmathsetmacro{\block}{3} % mini-batch size
       \pgfmathsetmacro{\ts}{2} % time steps
      
       \pgfmathsetmacro{\hht}{2}  % hidden layer height
       \pgfmathsetmacro{\iht}{4} % input and output layer height
       \pgfmathsetmacro{\wid}{0.5} % layer width
       
       \pgfmathsetmacro{\ihd}{1} % distance between input and hidden layer
       \pgfmathsetmacro{\tw}{2} % width between time steps
       
       \pgfmathsetmacro{\th}{\hht + \ihd + \iht / 2 - \hht / 2} % height between time steps
       
       \pgfmathsetmacro{\ar}{\wid * 1.5} % length of input and output arrow
       \pgfmathsetmacro{\iap}{4 / 5} % position of input arrow
       \pgfmathsetmacro{\oa}{8} % num of output arrow
       \pgfmathsetmacro{\oad}{\iht / \oa} % distance between output arrow
       
       \pgfmathsetmacro{\wtd}{.2} % distances between weight lines
        
       \pgfmathsetmacro{\steps}{\ts + \block - 1}
       \foreach \t in {1,...,\steps} {
         \pgfmathsetmacro{\x}{\t*(\wid + \tw)}
         \pgfmathsetmacro{\h}{\t*\th}
         
         \draw (\x, \h) rectangle +(\wid, \hht);
         \pgfmathsetmacro{\y}{\steps - \t + 1}
         
         \pgfmathparse{(\y==0)?1:0}
         \ifdim\pgfmathresult pt>0pt {%
           \path (\x, \h) --  +(\wid, 0) node[midway, below] {$y(t)$};
         } \else {%
           \path (\x, \h) --  +(\wid, 0) node[midway, below] {$y(t-\pgfmathprintnumber{\y})$};
         } %
         \fi
         
         \pgfmathsetmacro{\h}{\t*\th + \hht + \ihd}
         \draw (\x, \h) rectangle +(\wid, \iht);
         \pgfmathsetmacro{\y}{\steps - \t}
         
         \pgfmathparse{(\y==0)?1:0}
         \ifdim\pgfmathresult pt>0pt {%
           \path (\x, {\h + \iht}) --  +(\wid, 0) node[midway, above] {$x(t)$};
         } \else {%
           \path (\x, {\h + \iht}) --  +(\wid, 0) node[midway, above] {$x(t-\pgfmathprintnumber{\y})$};
         } %
         \fi
         \draw ({\x - \ar}, {\h + (\iap * \iht)}) -- +(\ar, 0);

         \draw ({\x + \wid}, {\h + \iht / 2}) -- node[midway, above]{$w_{ji}$} +(\tw, 0);
         \draw ({\x + \wid + \tw}, {\h + \iht / 2 - \wtd}) [dashed] -- +(-\tw, 0);
         \draw ({\x + \wid}, {\t*\th + 3*\hht / 4}) -- node[midway, left]{$w_{js}$} +(\tw, {\hht / 4 + \ihd + (\iht / 2 - \hht / 4)});
         \draw ({\x + \wid + \tw}, {\t*\th + \hht / 2 + \hht / 2 + \ihd + (\iht / 2 - \hht / 2) + \hht / 4 - \wtd}) [dashdotted] -- ({\x + \wid}, {\t*\th + 3*\hht / 4 - \wtd});
       
         \pgfmathparse{(\t>\steps-\block)?1:0}
         \ifdim\pgfmathresult pt>0pt {%
          \pgfmathsetmacro{\x}{(\t + 1)*(\wid + \tw)}
          \pgfmathsetmacro{\h}{(\t + 1)*\th}
          \draw (\x, \h) rectangle +(\wid, \hht);
        
          \pgfmathsetmacro{\x}{(\t + 1)*(\wid + \tw) + \wid}
          \pgfmathsetmacro{\h}{(\t + 1)*\th + \hht / 2}
          \draw (\x, \h) -- +(\tw, 0) node[midway, above] {$w_{kj}$};
          \draw ({\x + \tw}, {\h - \wtd}) [dotted] -- +(-\tw, 0);
         
          \pgfmathsetmacro{\x}{(\t + 1)*(\wid + \tw) + \wid + \tw}
          \pgfmathsetmacro{\h}{(\t + 1)*\th + \hht / 2 - \iht / 2}
          \draw (\x, \h) rectangle +(\wid, \iht);
          
          \pgfmathsetmacro{\y}{\steps - \t}
          \pgfmathparse{(\y==0)?1:0}
          \ifdim\pgfmathresult pt>0pt {%
            \path (\x, \h) --  +(\wid, 0) node[midway, below] {$z(t)$};
          } \else {%
            \path (\x, \h) --  +(\wid, 0) node[midway, below] {$z(t-\pgfmathprintnumber{\y})$};
          } %
          \fi
          
          \pgfmathsetmacro{\x}{(\t + 1)*(\wid + \tw) + \wid + \tw + \wid}
          \foreach \i in {1,...,\oa} {
            \draw (\x, {\h + 0.5 * \oad + (\i - 1) * \oad}) -- +(\ar, 0);
          } 
         } %
         \fi
       }
    \end{scope}  
\end{tikzpicture}


  
  \caption{Mini-Batch BPTT training, with mini-batch size 3 and 2 time steps back}
  \label{fig:bptt_block}
\end{center}
\end{figure}


\bibliographystyle{plain}
\bibliography{BP}

\end{document} 
